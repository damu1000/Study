\documentclass[letterpaper,twocolumn,10pt]{article}
\usepackage{epsfig,xspace,url}
\usepackage{authblk}


\title{Paper Summary\\
HA 2.a\\}
\author{Damodar Sahasrabudhe}
\affil{School of Computing, University of Utah}

\begin{document}

\maketitle
\section{Title of paper: P4: Programming Protocol-Independent Packet Processors}
Paper discussed in this summary is "P4: Programming Protocol-Independent Packet Processors"\cite{paper}.

\subsection{First pass information}
\label{sec:first}

\begin{enumerate}

\item {\it Category:} Paper suggests a new Programming language and presents examples.

\item {\it Context:} Paper is in area of SDN. Refers to multiple other papers such as - Kangaroo \cite{Kangaroo}, Portland\cite{Kangaroo}, NOSIX\cite{Kangaroo} etc.

\item {\it Assumptions:}  
\begin{itemize}
\item Actions will be built with protocol independent elements.
\item For sake of explanation, paper assumes there will be no packet processing during configuration.
\item Underlying switch has capability to parse the header.
\item Compilers targeting every type of underlying hardware will be easily available.

\end{itemize}

\item {\it Contributions:} Paper suggests building of an programming language to hide underlying hardware and protocols and provide simple and uniform interface to programmers, where controllers could be easily programmed.

%\item {\it Clarity:} Paper is written very well, code snapshots help to understand concepts quickly.

\end{enumerate}

\subsection{Second pass information}
\label{sec:second}



{\it Summary:} 

\begin{itemize}
\item Introduction: Over the period of time open flow headers are becoming more and more complex. It shows need to have flexible headers and need to add / remove header elements dynamically. This in turn needs more abstract programming interface which will be platform and protocol independent.

\item Abstract Forwarding Model: Improving further upon concepts of OpenFlow, switch could be have inbuilt parser tables (to parse header fields), parallel match-action tables (against serialized operations of OpenFlow tables) and action set.

\item Programming Language and Examples: Parallel operations of flow tables requires compiler should prepare Table Dependency Graph (TDG) and then map TDG to specific switch. P4 introduces new constructs for - header, parser, table, action and control. Each of them provides consistent way of configuring controller and/or switches irrespective of protocol or underlying hardware.

\end{itemize}

\subsection{Third pass information}
\label{sec:third}

\begin{itemize}

\item {\it Strengths:} 
\begin{itemize}
\item The idea, if implemented effectively, will make life of administrators easier.
\item It will be easy to use P4 and write and application to gather statistics and provide Bird's eye view.
\item Snapshots of code makes it easier to understand the paper.
\item Parallel processing of packets will give better performance than OpenFlow switch.

\end{itemize}


\item {\it Weaknesses:} 
\begin{itemize}
\item Packet tampering may cross boundaries of network protocol layers.
\item Seems over complex to be implemented and tested.

\end{itemize}


\item {\it Questions:} 
Is it practically possible to have compilers targeting so many protocols and underlying hardware?

\item {\it Interesting citations:} Scalable, high performance ethernet forwarding with CuckooSwitch

\item {\it Possible improvements:} 
\begin{itemize}
\item No need of multiple match action tables. Single entry can use AND / OR conditions to check complex conditions.
\end{itemize}

\item {\it Future work:} 
There should be some governing body to control custom protocols \ headers, otherwise there will be explosion of such private changes.
\end{itemize}

{
%\footnotesize 
\small 
\bibliographystyle{acm}
\begin{thebibliography}{1}
\bibitem{paper} P4: Programming Protocol-Independent Packet Processors

\bibitem{Kangaroo} C. Kozanitis, J. Huber, S. Singh, and G. Varghese, “Leaping multiple headers in a single bound: Wire-speed parsing using the Kangaroo system,” in IEEE INFOCOM, pp. 830–838, 2010
\bibitem{Portland}  R. Niranjan Mysore, A. Pamboris, N. Farrington, N. Huang, P. Miri, S. Radhakrishnan, V. Subramanya, and A. Vahdat, “PortLand: A scalable fault-tolerant layer 2 data center network fabric,” in ACM SIGCOMM, pp. 39–50, Aug. 2009.
\bibitem{NOSIX}  M. Raju, A. Wundsam, and M. Yu, “NOSIX: A lightweight portability layer for the SDN OS,” ACM SIGCOMM Computer Communications Review, 2014

\end{thebibliography}
\end{document}
}





