\documentclass[letterpaper,twocolumn,10pt]{article}
\usepackage{epsfig,xspace,url}
\usepackage{authblk}


\title{CS 6480: Class discussion summary\\
HA 3.b\\}
\author{Damodar Sahasrabudhe}
\affil{School of Computing, University of Utah}

\begin{document}

\maketitle
\section*{Discussion summary}

\begin{itemize}

\item {\it Summary:} 
Class discussed the main concept of the paper, its implementations and about strengths and weakness. Here few points for which I got more clarity:
\begin{itemize}
\item Programmers can more easily define own network protocols.
\item key challenge in implementation is  to achieve balance between expressiveness and ease of implementation.
\item Paper suggested 3 generalizations over openflow - reconfigurability, protocol independence and platform independence.
\item Performance will be a concern while operating at larger scale.
\item Example of table dependency - a scenario where next rule is set on MAC address. This rule is dependent upon finding out of MAC using ARP and can not be run unless MAC is found.
\item Example of L2 network deployment was more clarified after discussion - Top of the rack switch has  pre-populated destination ips/macs populated by control protocol header. Same  are populated in P4 header (say up1, up2 etc.) , P4/ openflow does not bother about how those get populated in Top of the rack switch.

\end{itemize} 

\item {\it Strengths:} 
\begin{itemize}
\item A good generalization of Openflow- configuring and maintaining switches will be easier.
\item Tremendous flexibility offered to programmers in terms of defining protocols.
\item Functions such as setfield are similar to C, making it easier to understand.
\end{itemize}

\item {\it Weakness:} 
\begin{itemize}
\item Some references in paper are left unexplained - Bridge Id in Dependency Graph, Ethertype in header.

\end{itemize}


\item {\it Connection with other work:} 
A brief mention of Portland\cite{Portland}, Click\cite{Click}.

\item {\it Future work:} was not discussed due to time constrain.

\end{itemize}

\begin{thebibliography}{1}
\bibitem{Portland} R. Niranjan Mysore, A. Pamboris, N. Farrington, N. Huang, P. Miri, S. Radhakrishnan, V. Subramanya, and A. Vahdat, “PortLand: A scalable fault-tolerant layer 2 data center network fabric,” in ACM SIGCOMM, pp. 39–50, Aug. 2009.
\bibitem{Click} E. Kohler, R. Morris, B. Chen, J. Jannotti, and M. F. Kaashoek, “The Click modular router,” ACM Transactions on Computer Systems, vol. 18, pp. 263–297, Aug. 2000.

\end{thebibliography}

\end{document}



