\documentclass[letterpaper,twocolumn,10pt]{article}
\usepackage{epsfig,xspace,url}
\usepackage{authblk}


\title{CS 6480: Paper reading summary\\
HA 1.a\\}
\author{Your Name}
\affil{School of Computing, University of Utah}

\begin{document}

\maketitle

\section*{Notes}

{\it 
Please keep the following in mind:

\begin{itemize}

\item This template is to be completed for each assigned paper based on your personal
reading of the paper.\footnote{The structure of this paper summary report template was derived from the 
three-pass paper reading method described by S. Keshav~\cite{how.to.read}.}

\item I expect the complete paper summary, i.e., all sections in this template combined, to be one to two pages long.

\item Make sure that in your summary you address each of the sections and items.

\item When you reference papers, including the paper you are summarizing, that should
be done with proper citations. (Avoid sentences like: ``This is a summary for~\cite{openflow.wp}''.
Rather say something like: ``This writeup provides a summary of the OpenFlow white paper
by McKeown et.al.~\cite{openflow.wp}".)

\item {\bf Use Latex for your summaries.} It is a bit more work to get going, but
produce the correct results.

\item While the ``summarizing the facts'' parts are an important basis for
a good paper summary, the parts that require you to think about the paper
and make connections with other work is even more important. As such
Section~\ref{sec:third} will be weighted more heavily in terms
of grading. Specifically, summaries will be graded as follows:
\begin{itemize}

\item Section~\ref{sec:first}: 20\%.

\item Section~\ref{sec:second}: 20\%

\item Section~\ref{sec:third}: 60\%

\end{itemize}
For grading I will basically assume that your start with 100\% and then subtract
points for sections that are not sufficiently addressed.

\end{itemize}
}

\section*{Paper summary}

Replicate sections below for each paper read (if we read more than one paper for
a particular class).

\section{Title of paper: e.g., OpenFlow: Enabling Innovation in Campus Networks}

Paper discussed in this summary is ``OpenFlow: Enabling Innovation in Campus Networks''~\cite{openflow.wp}.

\subsection{First pass information}
\label{sec:first}

\begin{enumerate}

\item {\it Category:} What type of paper is this? (E.g., a measurement paper, analysis of an existing system, a description of a research prototype, etc.)

\item {\it Context:} In what technical area is the work described in the paper in? What other papers does it relate to? 

\item {\it Assumptions:}  What assumptions do the authors make? Do the assumptions appear to be valid? Why or why not?

\item {\it Contributions:} What are the paper's (stated or claimed) main contributions?

%\item {\it Clarity:} Is the paper well written?

\end{enumerate}

\subsection{Second pass information}
\label{sec:second}

\begin{itemize}

\item {\it Summary:} Summarize the paper in your own words. You want to capture
the problem being addressed, the essence of the technical approach used to address
the problem, plus a brief summary (at least a sentence) of each section of the paper.
(This part is {\bf not} about your opinion on whether the paper/problem/solution was
good or bad. You are factually capturing details about the paper.)

\end{itemize}

\subsection{Third pass information}
\label{sec:third}

In this section we move from factual aspects of the paper to
your opinion about the paper.
We are not reviewing the papers we read, but trying
to learn from them and find interesting directions
for future research. As such the key things to get out
of this pass are:

\begin{itemize}

\item {\it Strengths:} What are the strengths of the paper? 

\item {\it Weaknesses:} What are the weaknesses of the paper? 

\item {\it Questions:} What questions do you have about the paper? I.e., are there
parts that you did not fully understand? Or, parts that you find
questionable?

\item {\it Interesting citations:} References that interested you, either because
they describe some earlier work that you are not familiar with,
or because they seem to describe something that is of interest
for further exploration. Note any such references. (Make sure you
use proper citations for these.)

\item {\it Possible improvements:} Any ways in which the work described in
the paper could be
improved, or ways in which the problem definition being addressed
could be improved. (I.e., did they address the right problem).

\item {\it Future work:} Any new ideas for future work that was inspired by
your reading of the paper.

\end{itemize}

{
  %\footnotesize 
  \small 
  \bibliographystyle{acm}
  \bibliography{biblio}
}
\end{document}



